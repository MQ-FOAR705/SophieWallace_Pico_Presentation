%From https://egu2018.eu/PICO_how-to_guide_to_PICO.pdf
%Abstracted and templated by Brian Ballsun-Stanton, Macquarie University.
%original template by https://github.com/snowtechblog/pico-latex-presentation by Anselm Köhler

\documentclass[unknownkeysallowed,usepdftitle=false, parskip=full]{beamer}
% unknownkeysallowed is needed for mac and the newer latex version -> is more picky than before...
\usetheme[headheight=1cm,footheight=2cm]{boxes}
%\usetheme{default}


\usepackage{default}
\usepackage{graphicx}
%example pictures created via: http://lorempixel.com/1200/800/cats/Figure2/. Credit to http://lorempixel.com/images.php

\usepackage{epsfig}
\usepackage{siunitx}
\usepackage{color}
\usepackage{ifthen}
%usepackage{ragged2e}

\usepackage[T1]{fontenc}
\usepackage[utf8]{inputenc}
%https://tex.stackexchange.com/a/203804/5483

\usepackage[activate={true,nocompatibility},final,tracking=true,kerning=true,spacing=true,factor=1100,stretch=10,shrink=10]{microtype} % http://www.khirevich.com/latex/microtype/
\microtypecontext{spacing=nonfrench}

\usepackage{lipsum} % for dummy text only
\usepackage[UKenglish]{babel} %https://tex.stackexchange.com/a/27743 
\usepackage[pangram]{blindtext} % https://tex.stackexchange.com/a/48411

%\usepackage{parskip} % from https://tex.stackexchange.com/q/11622
%\setlength{\parskip}{12pt} 

%\setparsizes{\parindent}{12pt}{\parfillskip}

%\usepackage{etoolbox} % as per https://tex.stackexchange.com/a/24331
%\appto\chapterheadendvskip{\vspace{-1\parskip}}
%\setparsizes{\parindent}{50pt plus 20pt minus 30pt}{\parfillskip}

\setbeamertemplate{navigation symbols}{}%remove navigation symbols
\setbeamersize{text margin left=1cm,text margin right=1cm}

% some colors
\definecolor{grau}{gray}{.5}
\definecolor{slfcolor}{rgb}{0,0.6274,0.8353}
\definecolor{wslcolor}{rgb}{0,0.4,0.4}

% setup links
\hypersetup{%
	%linkbordercolor=green,%
	colorlinks=false,%
	pdfborderstyle={/S/U/W 0},%
	%pdfpagemode=FullScreen,%
	pdfstartpage=4%
	}

% setup some fonts
\setbeamerfont{title}{series=\bfseries, size=\small}
\setbeamerfont{author}{size*={5pt}{0pt}}
\setbeamerfont{institute}{size*={3pt}{0pt}}
\setbeamerfont{bodytext}{size=\scriptsize}
	
% Title setup	
\title{Managing Qualitative Data In The Field: \\ Automation, Images and Metadata}
\author{Sophie Wallace\inst{1} (\texttt{sophie.wallace@students.mq.edu.au}) \inst{1}}
\institute{\inst{1}Macquarie University, Sydney, NSW}

% add title in headbox
\setbeamertemplate{headline}
{\leavevmode
\begin{beamercolorbox}[width=1\paperwidth]{head title}
  % LOGO
  \begin{columns}[t, totalwidth=\textwidth]
  \begin{column}[c]{1.05cm}
   \includegraphics[width=1cm]{Tropy_Icon.jpg}
  \end{column}
  % TITLE
   \begin{column}[c]{10.6cm}
   \centering \usebeamerfont{title} \textcolor{slfcolor}{\inserttitle} \\
   \centering \usebeamerfont{author} \color[rgb]{0,0,0} \insertauthor \\
   \vspace{-0.05cm}
   \centering \usebeamerfont{institute} \insertinstitute
  \end{column}
  % PICTURE
  \begin{column}[c]{1.15cm}
    \hspace{0.005cm}
    \includegraphics[width=1cm]{ExifTool_Icon.jpg}
  \end{column}
  \end{columns}
  {\color{slfcolor}\hrule height 1pt\vspace{0.1cm}}
\end{beamercolorbox}%
}

% setup the navigation in footbox
% first set some button colors
\newcommand{\buttonactive}{\setbeamercolor{button}{bg=wslcolor,fg=white}}
\newcommand{\buttonpassive}{\setbeamercolor{button}{bg=slfcolor,fg=black}}
% now set up that the one active one gets the new color.
\newcommand{\secvariable}{nothing}
% therefore we write before each section (well, everything which should be part of the navi bar)
% the variable \secvariable to any name which is in the next function ...
\newcommand{\mysection}[1]{\renewcommand{\secvariable}{#1}
}
% ... compaired to strings in the following navibar definition ...
\newcommand{\tocbuttoncolor}[1]{%
 \ifthenelse{\equal{\secvariable}{#1}}{%
    \buttonactive}{%
    \buttonpassive}
 }
% ... here we start to set up the navibar. each entry is calling first the function \tocbuttoncolor with the argument which should be tested for beeing active. if active, then change color. afterwards the button is draw. so to change that, you need to change the argument in \toc..color, the first in \hyperlink and before each frames definition... A bit messed up, but works...
\newlength{\buttonspacingfootline}
\setlength{\buttonspacingfootline}{-0.2cm}
\setbeamertemplate{footline}
{\leavevmode
\begin{beamercolorbox}[width=1\paperwidth]{head title}
  {\color{slfcolor}\hrule height 1pt}
  \vspace{0.05cm}
  % set up the buttons in an mbox
  \centering \mbox{
    \tocbuttoncolor{abstract}
    \hyperlink{abstract}{\beamerbutton{2 Minute Madness}}
    \tocbuttoncolor{radar}
    \hspace{\buttonspacingfootline}
      \hyperlink{radar}{\beamerbutton{Section 1}}

    \tocbuttoncolor{line}
    \hspace{\buttonspacingfootline}
      \hyperlink{line}{\beamerbutton{Section 2}}
    \tocbuttoncolor{major}
    \hspace{\buttonspacingfootline}
      \hyperlink{major}{\beamerbutton{Section 3}}
    \tocbuttoncolor{slab}
    \hspace{\buttonspacingfootline}
      \hyperlink{slab}{\beamerbutton{Section 4}}
    \tocbuttoncolor{minor}
    \hspace{\buttonspacingfootline}
      \hyperlink{minor}{\beamerbutton{Section 5}}
    \tocbuttoncolor{conclusion}
    \hspace{\buttonspacingfootline}
      \hyperlink{conclusion}{\beamerbutton{Conclusion}}
    % this last one should normaly not be used... it will open the preferences to change the 
    % behaviour of the acrobat reader in fullscreen -> usefull in pico...
    \setbeamercolor{button}{bg=white,fg=black}
    % for presentation
    %\hspace{-0.1cm}\Acrobatmenu{FullScreenPrefs}{\beamerbutton{\#}}
    % for upload
    
     
\Acrobatmenu{FullScreenPrefs}{\vspace{0.3cm}\hspace{0.24cm}\mbox{%
      \includegraphics[height=0.04\textheight,keepaspectratio]{%
	  figure/CreativeCommons_Attribution_License.eps}%
	  }}
   }
    \vspace{0.05cm}
\end{beamercolorbox}%
}


\begin{document}


%%%%%%%%%%%%%%%%%%%%%%%%%%%%%%%%%%%%%%%%%%%%%%%%%%%%%%%%%%%%%%%%%%%%%%%%%%
\mysection{abstract}
%%%%%%%%%%%%%%%%%%%%%%%%%%%%%%%%%%%%%%%%%%%%%%%%%%%%%%%%%%%%%%%%%%%%%%%%%%
\begin{frame}\label{\secvariable}

\usebeamerfont{bodytext}

\parbox{\linewidth}{

%Original template: https://github.com/snowtechblog/pico-latex-presentation by Anselm Köhler

\textbf{Aim:} This research project aims to demonstrate a streamlined process of managing images on a Mac laptop. This includes exporting the images from the default Photo library, embedding metadata into the image, organising the images in a research photo management system and storing the images securely. 
\vspace{12pt}

\textbf{Why:} This it to make the research process more manageable and efficient. Messy and unorganised data is very common in the field. Using this workflow enables the researcher to sit back whilst the photos are organised and metadata is added to the images.
\vspace{12pt}

\textbf{Useful for:} This is useful for researchers and students in the field who take photographs as part of their data collection. 
\vspace{12pt}

\textbf{Significance:} This provides better time management, detail, efficiency and reduces the stress of managing large data sets. Additionally, securely storing the images prevents the loss of data and information. 
\vspace{12pt}

\textbf{Additional information:} Released under MIT License. Further details can be found on github at: $https://github.com/MQ-FOAR705/SophieWallace_Pico_Presentation$ and $https://github.com/MQ-FOAR705/SophieWallace_POC_Contents$

}
 
  
\end{frame}

\begin{frame}\label{\secvariable}
  \begin{columns}[t]
  %https://tex.stackexchange.com/a/7452/5483
  \begin{column}[c]{0.45\textwidth}
%http://lorempixel.com/1200/800/cats/Figure2/     
%http://lorempixel.com/1200/800/cats/Figure3/
\includegraphics[width=1\textwidth,height=0.5\textheight,keepaspectratio]{%
Automator_Demo.png}\\
\vspace{12pt}
\includegraphics[width=1\textwidth,height=0.5\textheight,keepaspectratio]{%
Tropy_Demo.png}
    \end{column}
    \begin{column}[c]{0.50\textwidth}
    \parbox{\linewidth}{
      
  
 \textbf{Image 1:} Automator workflow that exports images from the Mac Photos application and embeds metadata. An ExifTool script adds the name of the researcher, the original date and time  and prompts the user to write the location where the photos were taken. 
\vspace{8pt}

\textbf{Image 2:} How these images and their metadata are recorded in Tropy. Prior to this, the images were named generically and the other categories were empty. 
             
 
   
      }
    \end{column}
    
  \end{columns}

  
\end{frame}
%%%%%%%%%%%%%%%%%%%%%%%%%%%%%%%%%%%%%%%%%%%%%%%%%%%%%%%%%%%%%%%%%%%%%%%%%%
\mysection{radar}
%%%%%%%%%%%%%%%%%%%%%%%%%%%%%%%%%%%%%%%%%%%%%%%%%%%%%%%%%%%%%%%%%%%%%%%%%%
\begin{frame}\label{\secvariable}
  \begin{columns}[t]
  %https://tex.stackexchange.com/a/7452/5483
    \begin{column}[c]{0.50\textwidth}
    \parbox{\linewidth}{

\textbf{Using Automator:} 
Automator is an exclusive Mac application that programs a workflow by automating repetitive tasks using a drag and drop interface. $https://macosxautomation.com/automator/$   
      \vspace{8pt}
      
Workflows are created by combining a set of actions to produce a result. There are numerous variables and pathways to achieve this. A shell script can also be included in the workflow.
      }
    \end{column}
    \begin{column}[c]{0.45\textwidth}
%http://lorempixel.com/1200/800/cats/Figure2/     
%http://lorempixel.com/1200/800/cats/Figure3/
\includegraphics[width=1\textwidth,height=0.5\textheight,keepaspectratio]{%
figure/Automator1.png}\\
\vspace{12pt}
\includegraphics[width=1\textwidth,height=0.5\textheight,keepaspectratio]{%
figure/Automator2.png}
    \end{column}
  \end{columns}

  
\end{frame}

%%%%%%%%%%%%%%%%%%%%%%%%%%%%%%%%%%%%%%%%%%%%%%%%%%%%%%%%%%%%%%%%%%%%%%%%%%
\mysection{line}
%%%%%%%%%%%%%%%%%%%%%%%%%%%%%%%%%%%%%%%%%%%%%%%%%%%%%%%%%%%%%%%%%%%%%%%%%%
\begin{frame}\label{\secvariable}
\begin{center}
  \vspace{-0.5cm}
  %http://lorempixel.com/1200/800/cats/Figure4/
 \includegraphics[width=1\textwidth,height=0.75\textheight,keepaspectratio]{%
  figure/ExifTool1.png}
\end{center}
  \vspace{-0.5cm}
   \textbf{Using ExifTool:} ExifTool is an open-source software program for reading, writing, and manipulating image, audio, video, and PDF metadata. The image above shows ExifTool displaying metadata already embedded in an image.
\end{frame}

%%%%%%%%%%%%%%%%%%%%%%%%%%%%%%%%%%%%%%%%%%%%%%%%%%%%%%%%%%%%%%%%%%%%%%%%%%
\mysection{major}
%%%%%%%%%%%%%%%%%%%%%%%%%%%%%%%%%%%%%%%%%%%%%%%%%%%%%%%%%%%%%%%%%%%%%%%%%%
\begin{frame}\label{\secvariable} %%Eine Folie
\begin{center}
%http://lorempixel.com/1200/800/cats/Figure5/
\includegraphics[width=1\textwidth,height=0.8\textheight,keepaspectratio]{%
figure/ExifTool2.png}
\end{center}

    \parbox{\linewidth}{
\textbf{Combining Automator and ExifTool:} With the automator action "run shell script", a basic ExifTool script was able to be included directly in this workflow.  
}
\end{frame}

%%%%%%%%%%%%%%%%%%%%%%%%%%%%%%%%%%%%%%%%%%%%%%%%%%%%%%%%%%%%%%%%%%%%%%%%%%
\mysection{slab}
%%%%%%%%%%%%%%%%%%%%%%%%%%%%%%%%%%%%%%%%%%%%%%%%%%%%%%%%%%%%%%%%%%%%%%%%%%
\begin{frame}\label{\secvariable}
%http://lorempixel.com/1200/800/cats/Figure6/
\begin{center}
\includegraphics[width=1\textwidth,height=0.75\textheight,keepaspectratio]{%
figure/ExifTool2.png}
\end{center}
    \parbox{\linewidth}{

This shell script adds metadata to Tropy categories such as creator, name, date and type. Due to the arbitrary nature of 'type', a prompt was included to ask for geo-location.
}

\end{frame}


%%%%%%%%%%%%%%%%%%%%%%%%%%%%%%%%%%%%%%%%%%%%%%%%%%%%%%%%%%%%%%%%%%%%%%%%%%
\mysection{minor}
%%%%%%%%%%%%%%%%%%%%%%%%%%%%%%%%%%%%%%%%%%%%%%%%%%%%%%%%%%%%%%%%%%%%%%%%%%
\begin{frame}\label{\secvariable} %%Eine Folie
\begin{center}
%http://lorempixel.com/1200/800/cats/Figure7/
\includegraphics[width=1\textwidth,height=0.7\textheight,keepaspectratio]{%
figure/Tropy_Before.png}
\end{center}
\vspace{-0.2cm}

\textbf{Using Tropy:} Tropy is an open-source software program for research photo management. This tool imports images and organises them with metadata and tags. Metadata needs to be added manually. Above displays how generic images are displayed after importing. 

\end{frame}

%%%%%%%%%%%%%%%%%%%%%%%%%%%%%%%%%%%%%%%%%%%%%%%%%%%%%%%%%%%%%%%%%%%%%%%%%%
\mysection{minor}
%%%%%%%%%%%%%%%%%%%%%%%%%%%%%%%%%%%%%%%%%%%%%%%%%%%%%%%%%%%%%%%%%%%%%%%%%%
\begin{frame}\label{\secvariable} %%Eine Folie
\begin{center}
%http://lorempixel.com/1200/800/cats/Figure7/
\includegraphics[width=1\textwidth,height=0.7\textheight,keepaspectratio]{%
Tropy_Demo.png}
\end{center}
\vspace{-0.2cm}

%
\textbf{Combining tools:} With Automator, ExifTool and Tropy, the images are uploaded in a more streamlined process and metadata is added seamlessly along the way. The image above displays the final result.


\end{frame}

%%%%%%%%%%%%%%%%%%%%%%%%%%%%%%%%%%%%%%%%%%%%%%%%%%%%%%%%%%%%%%%%%%%%%%%%%%
\mysection{conclusion}
%%%%%%%%%%%%%%%%%%%%%%%%%%%%%%%%%%%%%%%%%%%%%%%%%%%%%%%%%%%%%%%%%%%%%%%%%%
\begin{frame}\label{\secvariable}

  More information on this presentation and related content can be found here:
    \begin{itemize}
   \item  \textbf{Pico Presentation:} 
   $https://github.com/MQ-FOAR705/SophieWallace_Pico_Presentation$
  \item \textbf{Proof of Concept, Images and Project Management:} $https://github.com/MQ-FOAR705/SophieWallace_POC_Contents$
   \end{itemize}

  \usebeamerfont{bodytext}
 

\end{frame}



\end{document}